%% GLOSARIO

\newglossaryentry{formafuerte}{name={\textbf{forma fuerte}},description={Forma que por combinaci\'on de rectas o de elementos angulares transmite tensi\'on al conjunto. Su contraposici\'on ser\'ia la forma suave que transmite fluidez a trav\'es de ella.}} 

\newglossaryentry{clavealta}{name={\textbf{clave alta}},description={Decimos de una imagen que est\'a en clave alta cuando el blanco o tonos muy altos son claramente presentes y dominantes en la imagen. Lo contrario es que estuviesen en \gls{clavebaja}}}

\newglossaryentry{clavebaja}{name={\textbf{clave baja}},description={Decimos de una imagen que est\'a en clave baja cuando es el negro el color dominante o bien la gamas m\'as oscuras del gris.}}

\newglossaryentry{pornografia}{name={\textbf{pornograf\'ia}},description={La pornograf\'ia cubre el espectro de im\'agenes concebidas y realizadas para provocar la excitaci\'on sexual tanto de hombres como de mujeres. Dado que lo que es considerado pornogr\'afico, es decir capaz de estimular sexualmente a un individuo, depende de los criterios morales de cada sociedad y tiempo, podemos encontrarnos con el caso de que una determinada imagen que se considera pornogr\'afica en un momento dado no lo sea pasado el tiempo o en otro lugar, ante otro espectador.}}

\newglossaryentry{angulovision}{name={\textbf{\'angulo de visi\'on}},description={Es el \'angulo que cubre una lente con una determinada dist\'ancia focal teniendo en cuenta las dimensiones del sensor de la c\'amara o de la pel\'icula empleada. Para una c\'amara ``full frame'' se pueden considerar los siguientes:}}


%% ACRONIMOS

\newacronym{ca}{CA}{Clave Alta}
\newacronym{cb}{CB}{Clave Baja}

