\chapter{Ejemplos de artistas del desnudo}

\section{Herb Ritts}
\index{Ritts, Herb}

\subsection{Biograf\'ia}

\subsection{Caracter\'isiticas de su obra}

Parece ser coincidente que la obra de Ritts se caracteriza por la imposici\'on de l\'ineas rectas limpias\cite{ciherbritts} y \gls{formafuerte}

\subsection{Obras destacadas}



\section{Robert Mapplethorpe}
\index{Mapplethorpe, Robert}

\subsection{Biograf\'ia}

\subsection{Caracter\'isiticas de su obra}


\subsection{Obras destacadas}

\section{Anton Belovodchenco}
\index{Belovodchenco, Anton}

\subsection{Biograf\'ia}

\subsection{Caracter\'isiticas de su obra}
Belovodchenco tiene un acercamiento escultural cl\'asico\cite{ciantonbelovodchenco} en el modo en el que muestra los cuerpos desnudos. En sus im\'agenes juega con los matices de la luz, la escala de grises, para destacar las peculiaridades \'unicas del cuerpo humano, as\'i como el juego de rectas y curvas en la composici\'on que trasciende a veces la mera est\'etica corporal para adentrarse en el paisaje del cuerpo humano.



\subsection{Obras destacadas}


\section{James Houston}
\index{Houston, James}

\subsection{Biograf\'ia}
Tal y como podemos leer en su informaci\'on biogr\'afica\cite{jameshoustonweb} su formaci\'on como escultur ayud\'o e influy\'o en el modo en el que ha desarrollado su estilo art\'istico. No hay mucha m\'as informaci\'on al respecto.

\subsection{Caracter\'isiticas de su obra}

En el art\'iculo indicado en la bibliograf\'ia \cite{cijameshouston} puede verse, en las im\'agenes adjuntas las composiciones solo alcanzables a bailarines y bailarinas que conjuntan la belleza del cuerpo con la de la pose, as\'i como el uso de la \acrshort{cb} (\gls{clavebaja}) en sus im\'agenes.

Como ya se ha indicado en su biograf\'ia, sus im\'agenes est\'an muy influenciadas por la est\'etica cl\'asica.


\subsection{Obras destacadas}



